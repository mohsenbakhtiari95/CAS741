\documentclass{article}

\usepackage{amsmath, mathtools}
\usepackage{amsfonts}
\usepackage{amssymb}
\usepackage{graphicx}
\usepackage{colortbl}
\usepackage{xr}
\usepackage{hyperref}
\usepackage{longtable}
\usepackage{xfrac}
\usepackage{tabularx}
\usepackage{float}
\usepackage{siunitx}
\usepackage{booktabs}
\usepackage{caption}
\usepackage{pdflscape}
\usepackage{afterpage}

\hypersetup{
    colorlinks=true,
    linkcolor=red,
    filecolor=magenta,
    urlcolor=cyan
}

\title{Problem Statement and Goals\\Integrated Multi-Physics Simulation Software for Porous Media}
\author{Mohsen Bakhtiari\\PhD Candidate, Civil Engineering\\McMaster University}
\date{\today}

\begin{document}

\maketitle

\begin{table}[H]
\caption{Revision History}
\label{TblRevisionHistory}
\begin{tabularx}{\textwidth}{llX}
\toprule
\textbf{Date} & \textbf{Developer(s)} & \textbf{Change}\\
\midrule
18 January 2026 & Mohsen Bakhtiari & Initial release of document\\
\bottomrule
\end{tabularx}
\end{table}

\section{Problem Statement}

Understanding and predicting the long-term behavior of porous materials subjected to coupled hydro-thermal-chemical-mechanical processes is a critical challenge in many engineering disciplines, including civil, petroleum, geotechnical, and environmental engineering. Phenomena such as multi-phase fluid flow, heat transfer, mineral transport, precipitation and dissolution, and mechanically induced damage are strongly interdependent and evolve over time in a highly nonlinear manner.

Although advanced numerical models exist in research contexts, many available tools are limited to single-physics problems, lack extensibility, or are not designed for flexible research-oriented experimentation. In particular, integrating multi-phase flow, chemical transport, thermal effects, and mechanical damage—including localized discontinuities such as cracks and material interfaces—within a unified and reusable software framework remains challenging. As a result, researchers often rely on fragmented scripts or ad-hoc implementations that are difficult to maintain, validate, and extend.

This project aims to address these limitations by transforming an existing research-grade MATLAB codebase into a structured, modular, and user-oriented software platform for coupled hydro-thermal-chemical-mechanical analysis of porous media.

\subsection{Problem}

The primary challenge is to design and implement a software system capable of reliably simulating coupled multi-physics processes in porous materials while remaining flexible, extensible, and suitable for research and engineering applications. Key difficulties include:

\begin{itemize}
    \item Managing strong coupling between fluid flow, heat transfer, chemical transport, and mechanical response.
    \item Supporting multi-phase flow (liquid and vapor) with both advective and diffusive transport mechanisms.
    \item Accurately modeling time-dependent mineral precipitation and dissolution and their effects on material properties.
    \item Capturing mechanical damage and localized discontinuities such as cracks, faults, and material interfaces.
    \item Ensuring numerical robustness through verification, validation, and sensitivity analysis.
\end{itemize}

Existing solutions often lack a unified architecture that supports all these capabilities simultaneously. This project seeks to develop a coherent software framework that integrates these components and enables realistic simulation of porous-media behavior under diverse environmental and loading conditions.

\subsection{Inputs and Outputs}

\textbf{Inputs.}  
The software will accept a range of user-defined inputs, including:
\begin{itemize}
    \item Geometric descriptions and finite element meshes.
    \item Material properties such as porosity, permeability, thermal conductivity, and mechanical parameters.
    \item Initial and boundary conditions for fluid pressures, saturations, temperature, and chemical concentrations.
    \item Transport and reaction parameters governing mineral precipitation and dissolution.
    \item Mechanical loading conditions and constitutive model parameters.
    \item Time-stepping and numerical control settings.
\end{itemize}

\textbf{Outputs.}  
The software will produce outputs including:
\begin{itemize}
    \item Spatial and temporal distributions of fluid pressures and saturations.
    \item Temperature and chemical concentration fields.
    \item Evolution of mineral content due to precipitation and dissolution.
    \item Stress, strain, and damage fields, including crack initiation and propagation.
    \item Diagnostic data for convergence, validation, and sensitivity analyses.
    \item Visualization-ready results for post-processing and interpretation.
\end{itemize}

\subsection{Stakeholders}

Potential stakeholders include:
\begin{itemize}
    \item Academic researchers studying porous media and multi-physics phenomena.
    \item Graduate students conducting advanced numerical and computational research.
    \item Engineering practitioners in civil, geotechnical, petroleum, mining, and environmental engineering.
    \item Conservation engineers assessing durability and rehabilitation of masonry and historical structures.
    \item Computational engineers and software developers extending scientific simulation tools.
\end{itemize}

\subsection{Environment}

The software will be designed to run on common research computing platforms, including Windows, macOS, and Linux-based systems. The implementation will emphasize modularity and portability to support future extensions, including graphical interfaces and high-performance computing workflows.

\section{Goals}
\begin{enumerate}
    \item Develop a modular and maintainable software architecture for coupled hydro-thermal-chemical-mechanical simulations.
    \item Enable robust simulation of multi-phase flow, heat transfer, chemical transport, and mechanical damage within a unified framework.
    \item Provide a flexible research tool supporting verification, validation, and sensitivity studies.
    \item Improve usability and extensibility compared to standalone research scripts.
\end{enumerate}

\section{Stretch Goals}
\begin{enumerate}
    \item Implement advanced visualization and post-processing capabilities.
    \item Extend the framework to support additional constitutive and coupling models.
    \item Explore integration with graphical user interfaces and high-performance computing environments.
    \item Apply the software to benchmark problems and real-world engineering case studies.
\end{enumerate}

\end{document}
